%! TEX root = **/000-main.tex
% vim: spell spelllang=en:

\section{Implementation}%
\label{sec:implementation}

The main result of this project is the program neuropy\cite{neuropy}.
The purpose of the program is to generate visualizations of the Hodgkin-Huxley model on neuron geometries utilizing extrapolation methods to minimize the impact of performance issues.
The code has two main steps; importing simulation data and generating the visualizations.
Since generating the simulation data on neuron geometries isn't the focus of the project, neuropy requires this information to be given in input files.
This can be done using Neuro-VISOR, but neuropy will take any pair of ``.csv'' and ``.swc'' files with corresponding indices.
It is designed to make comparison of different extrapolation methods and contexts easy and straightforward. 
All three of the linear function definitions explained in \cref{sec:extrapolation} are implemented, and new extrapolation methods can be added by simply adding a new python function.


\subsection{Importing Data}%
\label{sub:importing_data}
The simulation values are obtained from Neuro-VISOR\cite{neuroVISOR} and are stored in the ``.csv'' format where each row corresponds to one solve step.
The Neuron geometries are stored in the ``.swc'' file format\cite{swc} and were obtained from NeuroMorpho\cite{neuroMorpho}.
It is important to note that the ``.swc'' file currently used in the code is actually obtained from Neuro-VISOR.
While Neuro-VISOR also obtains it's ``.swc'' files from NeuroMorpho but slightly alters them, changing the order of the node indexes\cite{neuropy}.
Neuro-VISOR doesn't provide these indices when exporting the simulation data so we are required to use it's neuron geometry files.

\subsection{Visualization}%
\label{sub:visualization}
There are multiple temporal parameters which should be properly disambiguated.
To begin, we assume that in the simulation data file each row is separated by a fixed time step.
This may be different than the time step of the solver, as is the case in Neuro-VISOR.
The most straight forward parameter in neuropy is the target fps, $f_t$, which is simply the frames per second (fps) of the output video.
This should be the first parameter chosen as the other's depend on it.
The next parameter is the rate of the simulation data, $f_d$ in fps.
Neuropy will read the next data point every $f_t / f_d$ frames.
$f_d$ should be a divisor of $f_t$ so none of the frames are hidden or removed.
Note that the playback speed will also be affected depending on $f_d / f_t$ and the size of the data's fixed time step.
Another parameter is the number of extrapolation points per $f_d$, denoted $f_e$ which determines how frequently the visualization is updated using the extrapolation values.
For example, if $f_t=60,f_d=1,f_e=10$ then every $6$ frames the visualization is updated.
Similar to $f_d$, $f_e$ should be a divisor of $f_t$ in order for the video to display every frame.
The last parameter is the number of data points skipped per read, $f_s$.
In other words, whenever a new data point is read it will skip the next $f_s - 1$ rows.
This is mainly used to easily create large differences between data points which can be useful for creating clear visualization examples.

\subsection{Results}%
\label{sub:results}
The main use of neuropy is to study how the extrapolation methods behave given various values for the parameters defined in \cref{sub:visualization}.
Neuropy includes $62$ examples with different parameter values and extrapolation methods.
However, due to the relationships between the parameters, the data, and the methods, it can be difficult to pick specific videos which demonstrate interesting differences.
To alleviate this we will list a few specific examples and what they demonstrate alongside the relevant file names in the neuropy repository\cite{neuropy}.
For all of the following examples, let $f_t=60$.
Also, if ``jump'' is $1$ the tangent line was used for extrapolation, otherwise the third option was used.

\begin{table}[H]
    \begin{center}
    \caption{basicComparison\_center.mp4}%
    \label{tab:exampleOne}
    \begin{tabular}{c|c|c|c}
        \toprule
        $f_d$ & $f_s$ & $f_e$ & jump \\
        \midrule
        1 & 5 & 0 & 0 \\
        1 & 5 & 15 & 1 \\
        1 & 5 & 30 & 0 \\
        1 & 5 & 30 & 1 \\
        \bottomrule
    \end{tabular}
    \end{center}

\end{table}

The four configurations in \cref{tab:exampleOne} are chosen to show the difference between no extrapolation and two extrapolations with different rates.
It also includes both of the extrapolation techniques.
$f_s$ is set to $5$ to make the four options more easily distinguishable without getting rid of too many data points.
The first configuration is equivalent to disabling extrapolation, or using naive extrapolation, and is used as the baseline for the other three configurations.
The last two configurations show the striking difference between the extrapolation methods, and the two configurations using the tangent line are only subtly  different.

\begin{table}[H]
    \begin{center}
    \caption{jumpComparison\_full.mp4}%
    \label{tab:exampleTwo}
    \begin{tabular}{c|c|c|c}
        \toprule
        $f_d$ & $f_s$ & $f_e$ & jump \\
        \midrule
        15 & 1 & 0 & 0 \\
        1 & 15 & 0 & 0 \\
        1 & 15 & 30 & 0 \\
        1 & 15 & 30 & 1 \\
        \bottomrule
    \end{tabular}
    \end{center}

\end{table}

The four configurations in \cref{tab:exampleTwo} are more similar than in \cref{tab:exampleOne}, they are specifically chosen to highlight how the third extrapolation option behaves with extreme data.
All but the first configuration has $f_s=5$, which is very large but not so large that there is only one data point per oscillation.
The first two are baseline configurations without extrapolation, the first shows every datapoint and the second shows every fifteenth.
These are useful for comparing the values of the third configuration with the actual data points.
The fourth configuration is the same as the third save the extrapolation method in order to show the difference between the methods.
The fourth configuration, while slightly more informative than the second configuration, is notably difficult to follow.
The third configuration is impressively clear, however it is generally less accurate than the fourth configuration.
When viewing the corresponding video on neuropy, the L1 norm is displayed each frame to make it easy to compare both the accuracy of the extrapolation methods as well as the more subjective ``feel".




